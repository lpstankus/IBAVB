%!TeX root=../paper.tex

\section{Introduction}
\label{sec:intro}

\IEEEPARstart{P}{ath} bundling has emerged as a prominent method for reducing visual clutter and highlighting key insights in visualizations of large graph embeddings and trail-sets. Modern path bundling techniques are capable of processing datasets with millions of paths (edges or trails) on consumer hardware, offering a powerful means of visualizing complex data\cite{lhuillier:2017:survey}. However, path bundling introduces distortion and overlap in the visualization. While these techniques can effectively reduce visual clutter, they also risk inducing incorrect interpretations or assumptions about the data. A significant issue associated with path bundling is path ambiguity, where the bundled drawing suggests connections not present in the underlying data.

Despite its prevalence, path ambiguity is not explicitly addressed by most existing bundling methods. Notable techniques such as FDEB\cite{holten:2009}, KDEEB\cite{hurter:2012}, and CUBu\cite{van:2016} all prioritize spatial proximity of paths when creating bundles. While this approach significantly reduces visual clutter, it potentially creates bundles with unrelated paths, exacerbating the ambiguity problem previously described. Grid-based techniques, such as Winding Roads\cite{lambert:2010}, SBEB\cite{ersoy:2011}, and GBEB\cite{cui:2008}, split paths into smaller bundles, which helps maintain path coherence to some extent, but can still introduce ambiguity to the visualization.

Confluent Drawings (CD)\cite{dickerson:2002} address a similar issue by computing both the layout and clustered drawing of a graph. These methods propose strict requirements for edge clustering on the graph structure, which drastically limits ambiguity. However, CDs cannot be applied to all graphs, are computationally expensive, and, because of the rigorous formulation, are less effective at reducing visual clutter than bundling techniques; making CD techniques unsuitable for visualizing large networks.

% Confluent Drawings (CD)\cite{dickerson:2002} address a similar issue by computing both the layout and clustered drawing of a graph. These methods propose strict requirements for edge clustering based on the network structure and, by design, do not introduce ambiguity to the drawing. However, this approach is computationally expensive and, because of the rigorous formulation, less effective at reducing visual clutter than bundling techniques; making CD techniques unsuitable for visualizing large networks.

% Edge-Path Bundling (EP)\cite{wallinger:2022} attempts to fit in the middle ground between standard bundling and CD techniques. EP creates bundles using network structure with looser rigidity compared to CD techniques, removing what they define as \emph{independent edge ambiguity} by design. However, EP approach introduces significant spatial distortion compared to the original drawing, which can introduce other types of ambiguities in the visualizations. The distortion also limits EP's effectiveness in applications where preserving the spatial information of embeddings is crucial, such as mobility data visualization.

Given the shortcomings on CD techniques, hybrid approaches between path bundling and CD emerged to fit the gap between them, providing less ambiguous bundling\cite{bach:2016,zheng:2021,wallinger:2022}. However, regardless of these advancements, there remains a critical gap in the field: the need for a method that effectively reduces visual clutter while simultaneously minimizing edge ambiguity, especially when the spatial embeddings encode crucial information. Current approaches often sacrifice one aspect for the other, leaving room for significant improvement in balancing these competing objectives.

% NOTE:
% Lucas: Edge-Path tries to fill this gap, but I think we can argue that it does not fit well here. Even though its whole proposition is to remove independent edge ambiguities by design, it performs kinda similar to other methods in terms of ambiguity (using their own metric); and much worse using our own ambiguity metric. Thus I think we can argue that no such method exists currently. Our proposition (as seen in later examples) heavily reduces ambiguities by sacrificing clutter reduction, but is still much much better than confluent drawings (all according to our metric).

In this paper, we propose a novel approach for reducing visual clutter with minimal path ambiguity. Our contributions are fourfold: First, we introduce a new image-based metric for quantifying ambiguity at a local level throughout the drawing, providing pixel-level ambiguity values. Second, we propose an extension applicable to any iterative bundling technique to minimize ambiguity, utilizing our metric to constrain ambiguous bundle formation without altering the underlying algorithm. Third, we present an implementation of our approach using the CUBu framework, demonstrating an ambiguity-avoiding bundling technique. Fourth, we provide an open-source reproducibility package containing the code and data used in this paper.

The remainder of this paper is organized as follows: Section \ref{sec:rw} reviews related work on edge bundling. Section \ref{sec:ambiguity} introduces our metric for quantifying ambiguity. Section \ref{sec:bundling} details our method for enhancing iterative bundling techniques with ambiguity avoidance. Section \ref{sec:experiments} presents applications of our bundling method. Section \ref{sec:discussion} analyzes the key aspects of our metric and method. Section \ref{sec:limitations} addresses limitations of our work. Finally, Section \ref{sec:conclusion} summarizes our findings and concludes the paper.
