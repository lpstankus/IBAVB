%!TeX root=../paper.tex

\section{Related Work}\label{sec:rw}

% lucas: Not sure if it is okay to start directly with these subsections or not

Building upon our contributions described in Section \ref{sec:intro}, we review existing literature on path bundling and ambiguity measurements and avoidance.

% I couldn't think of a way to properly split edge bundling from trial bundling (and consequently path
% bundling with the combination of both) that wouldn't make this paragraph too long.
\textbf{Path bundling:} Since the introduction of edge bundling by Holten\cite{holten:2006}, numerous techniques emerged to bundle both graph drawings and trail-sets\cite{lhuillier:2017:survey}. Edge and trail bundling techniques, which we can globally denote as path bundling, aim to group paths by their spatial similarity and, optionally, their data and general network structure. Path bundling can be done using control-meshes\cite{cui:2008}, quadtrees for spatial-partitioning\cite{lambert:2010}, force-directed algorithms\cite{holten:2009,nguyen:2012,selassie:2011}, and multilevel ink-saving clustering\cite{gansner:2011}. Along them, image-based techniques\cite{telea:2010,ersoy:2011,hurter:2012,bottger:2014,wu:2015,van:2016,lhuillier:2017:ffteb,zeng:2019} operate on the pixels of path to leverage the GPU for scalability. While these methods are the most efficient, being able to bundle datasets with millions of paths, they often sacrifice fine control over bundle formation for performance.

The current state-of-the-art bundling methods group paths almost exclusively on spatial proximity. As such, these methods are all affected by \emph{path ambiguity} issues. While the proper definition of path ambiguity varies in the literature, in essence, ambiguity arises when there is loss of information about the connectivity patterns of the network. When there is ambiguity, users can be misled by the visualization and perceive false adjacencies or incorrect relationships between nodes. This leads to misinterpretation and wrong assumptions about the underlying data and structures. There have been works that aim to mitigate ambiguity in bundled drawings, however ambiguity is a complicated issue. When designing a bundling algorithm, one needs to mitigate three opposing forces:

\begin{enumerate}
\item \textbf{Removal of visual clutter:} Visual complexity of the image should be kept at a minimum.
\item \textbf{Ambiguity-avoidance:} The user's understanding of the underlying connectivity should be preserved.
\item \textbf{Geometric closeness to the original input:} Distortion of the original data should not be too extreme.
\end{enumerate}

These forces interact in complex ways. For instance, if one takes \#1 to the extreme and bundles too extensively, the amount of distortion necessary would likely break \#3 and make it difficult to maintain \#2, as paths with drastically different connectivity would be bundled together, leading to false adjacencies. On the other hand, prioritizing \#2 or \#3 too extensively might lead to bundled drawings with excessive visual clutter. This tension highlights the fundamental challenge in designing effective bundling algorithms that must balance these competing factors.

% lucas: Complexity analysis here is actually NP-completeness, don't know if we should mention it.
\textbf{Confluent Drawings (CD):} Introduced by Dickerson \textit{et al.}\cite{dickerson:2002}, CDs do not tackle the exact same problem as edge bundling but face similar issues. The core idea behind CD is the strict use of the graph network to create confluent bundles, where only edges from $K_{n,m}$ subgraphs can be grouped. As from the original proposition, CDs are also crossing-free; therefore not every graph has a valid confluent drawing. Different CD techniques have been proposed from a theoretical point of view\cite{eppstein:2006,hui:2007,eppstein:2013}, with considerations on their mathematical formulation and complexity analysis. Other works propose algorithms for CD variations, typically tackling both the graph layout problem and the clusterization of edges\cite{dickerson:2002,hirsch:2007,eppstein:2007,honciuc:2009}; this differentiates them from path bundling, which requires a layout as input. The strict requirements from CDs virtually eliminate ambiguity; however, following the bundling balancing act, the strong ambiguity-avoidance implies they cannot reduce visual clutter as extensively as bundling techniques.

\textbf{Ambiguity-avoidance:} As most bundling techniques converged to bundle based on spatial proximity, some approaches took inspiration from CDs to move towards less ambiguous path bundling. Selassie \textit{et al.}\cite{selassie:2011} presented a modified FDEB algorithm, incorporating directionality and structural information to improve bundle formation. Luo \textit{et al.}\cite{luo:2012} used a combination of spatial-partitioning, congestion metrics, and deformation of edges in what they defined as ambiguity-free edge bundling. Nocaj and Brandes\cite{nocaj:2013} forced cohesive edge angles near endpoints and minimal bends to avoid ambiguity on undirected graphs and introduced confluent spiral drawings for directed graphs. Hybrid techniques between CD and path bundling also emerged. Most notably, Bach \textit{et al.}\cite{bach:2016} used power-graphs as a base and used the new hierarchy to both draw and bundle the graph. Their approach was later improved by Zheng \textit{et al.}\cite{zheng:2021} to create power-confluent drawings with minimal edge crossings. All these approaches were important advances towards less ambiguous bundling, but they still suffer drastically with visual clutter compared to other state-of-the-art bundling techniques.

Finally, Wallinger \textit{et al.}\cite{wallinger:2022} proposed Edge-Path bundling (EP), a hybrid technique that uses weighted paths for bundling. Edge-Path groups long edges along the shorted path made from smaller edges, grouping edges solely on existing connections. Their work also expands the concept of edge ambiguity by categorizing it into three types: \emph{independent-edge ambiguity}, when edges with independent endpoints are bundled; \emph{path endpoint ambiguity}, when subsequent node connections lead to misinterpretation on how far connections extend within the chain; and \emph{path crossing ambiguity}, when shallow crossing angles create misinterpretation of the connectivity patterns. According to their classification, EP focuses on eliminating independent edge ambiguities.

Building on EP's ambiguity definition, we further expand this concept. Previously, ambiguity has been analyzed from the perspective of graph structure, where the only source of truth is the network structure itself. However, when considering a layout, the spatial encoding typically carries information about the data and relationships. Layouts are also cautiously crafted, as even simple design decisions, such as the distance between nodes, can influence users' interpretations\cite{mcgrath:1996,fabrikant:2008}. We propose that path ambiguity should not be treated as a binary concept, where all false connections carry equal weight. Instead, we consider false connections between similar structures, such as individual nodes within two heavily linked clusters, as less ambiguous than those between entirely unrelated regions.

\textbf{Ambiguity quality metrics:} Together with visualization tools, there has been significant advances in measuring the quality of said visualizations. Several works emerged to measure various aspects of graph visualization, including edge congestion\cite{carpendale:2001}, entropy\cite{tufte:1983}, faithfulness\cite{nguyen:2013,nguyen:2017}, readability\cite{dunne:2009,dunne:2015}, and entropy\cite{sips:2009,tatu:2009}. Despite that some of these metrics indirectly measured ambiguity, few works have ambiguity as a focus. Edge-Path\cite{wallinger:2022} presents a graph quality metric for ambiguity based on faithfulness, using the proportion of true and false connections. AmbiguityVis\cite{wang:2016} introduces several metrics tailor-made to measure ambiguity and displays them in a heatmap for users, allowing them to check for misleading areas. However, both of them have strict focus on graph structure, failing to consider the spatial layout. We propose a new metric that considers the spatial layout, aiming to provide a more comprehensive view of ambiguity in path bundling.
